\documentclass{article}

\usepackage{amsmath}
\usepackage{amssymb}
\usepackage{dsfont}
\usepackage{enumitem}
\usepackage{amsthm} % to enable the proofs
\usepackage{graphicx} % to enable the graphics
\usepackage{mathtools}

\usepackage{bbm}

\usepackage[hidelinks]{hyperref}

% to be read by all users : how to number theorems and definitions : 
% done by A. Humeau on 4/10/2012
\newtheorem{theorem}{Theorem}[section]
\newtheorem{corollary}[theorem]{Corollary}
\newtheorem{proposition}[theorem]{Proposition}
\newtheorem{lemma}[theorem]{Lemma}

\newtheorem{definition}{Definition}[section]
\newtheorem{remark}{Remark}[section]
\newtheorem{example}{Example}[section]
\newtheorem{notation}{Notation}[section]

% end of environment addition

\usepackage{geometry}
\geometry{top=2cm, bottom=2cm, left=2cm, right=2cm}

\usepackage[utf8]{inputenc}

% to insert code in the report
\usepackage{listings}
\usepackage{xcolor}
\lstset { %
    language=C++,
    backgroundcolor=\color{white}, % set backgroundcolor
    basicstyle=\footnotesize,% basic font setting
}
%end of addition to insert code in the report

\bibliographystyle{plain}

\DeclareMathOperator*{\mysup}{sup}

\begin{document}

\title{Smile Interpolation}
\author{Alexandre Humeau}
\date{\today}
\maketitle

\section{Motivations}
Using a smile is useful to define the distribution of a financial underlying at a given date

\begin{theorem}[Breeden-Litzenberger formula]
	Let $S$ be a financial asset. Let $\phi$ be the probability density function of $S_T$. Then
	\begin{equation}
		\phi(K) = \frac{d^2C^T}{dK^2}
	\end{equation}
	where $C^T$ is the $T$-forward price of a call of strike $K$
\end{theorem}

\noindent Let us assume that we have a smooth parametrisation of the smile $K \mapsto \sigma(K)$ for a given maturity $T$.

\begin{proposition}
	With the above notations
	\begin{equation}
		\phi(K) = \frac{\partial ^2 C^T}{\partial K^2} + \sigma''(K) \frac{\partial C^T}{\partial \sigma} + \sigma'(K)^2 \frac{\partial^2 C^T}{\partial \sigma^2} + 2 \sigma'(K) \frac{\partial^2 C^T}{\partial \sigma \partial K} 
	\end{equation}
\end{proposition}

If we want the density function to be a $\mathcal{C}^1$ function of the strike $K$, $K\mapsto \sigma(K)$ has to be at least a $\mathcal{C}^3$ function

\section{Interpolation via a polynomial function of degree 5}
\subsection{Hermite Interpolation}

A polynomial function of degree 5 between 0 and 1 can be determined by the values in 0 and 1, the first derivative in 0 and 1 and the second derivative in 0 and 1.

\noindent We have 

\begin{equation}
	f(x) = f(0) H_0^0(x) + f(1) H_1^0(x) + f'(0) H_0^1(x) + f'(1) H_1^1(x) + f''(0) H_0^2(x) + f''(1) H_1^2(x)
\end{equation}

\noindent where

\begin{equation}
\begin{aligned}
	H_0^0(x) &= 1 - 10 x^3 + 15 x^4 - 6 x^5\\
	H_0^1(x) &= t - 6 x^3 + 8 x^4 - 3 x^5\\
	H_0^2(x) &= \frac{1}{2} \left(x^2 - 3 x^3 + 3 x^4 - x^5\right)\\
	H_1^0(x) &= 10x^3 - 15 x^4 + 6 x^5\\
	H_1^1(x) &= -4x^3 + 7 x^4 - 3x^5\\
	H_1^2(x) &= \frac{1}{2} \left(x^3 - 2x^4 + x^5\right)
\end{aligned}
\end{equation}

\subsection{Smile interpolation methodology}

We assume that the function $K \mapsto \sigma(K)$ is a piecewise Hermite-interpolated polynomial of degree as a function of $\ln\left(\frac{K}{F}\right)$, where is $F=F_{0,T}$ is the forward at time $0$ of the underlying asset.\\

\noindent Let us assume that we have the set of quoted volatilities

\begin{equation*}
	\mathcal{V} = \left\{(K_i, \sigma_i), i = 1,\dots, n\right\}
\end{equation*}

\end{document}
