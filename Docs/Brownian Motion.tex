\documentclass{article}

\usepackage{amsmath}
\usepackage{amssymb}
\usepackage{dsfont}
\usepackage{enumitem}
\usepackage{amsthm} % to enable the proofs
\usepackage{graphicx} % to enable the graphics
\usepackage{mathtools}

\usepackage{bbm}

\usepackage[hidelinks]{hyperref}

% to be read by all users : how to number theorems and definitions : 
% done by A. Humeau on 4/10/2012
\newtheorem{theorem}{Theorem}[section]
\newtheorem{corollary}[theorem]{Corollary}
\newtheorem{proposition}[theorem]{Proposition}
\newtheorem{lemma}[theorem]{Lemma}

\newtheorem{definition}{Definition}[section]
\newtheorem{remark}{Remark}[section]
\newtheorem{example}{Example}[section]
\newtheorem{notation}{Notation}[section]

% end of environment addition

\usepackage{geometry}
\geometry{top=2cm, bottom=2cm, left=2cm, right=2cm}

\usepackage[utf8]{inputenc}

% to insert code in the report
\usepackage{listings}
\usepackage{xcolor}
\lstset { %
    language=C++,
    backgroundcolor=\color{white}, % set backgroundcolor
    basicstyle=\footnotesize,% basic font setting
}
%end of addition to insert code in the report

\bibliographystyle{plain}

\DeclareMathOperator*{\mysup}{sup}

\begin{document}

\title{Brownian Motion calculation}
\author{Alexandre Humeau}
\date{\today}
\maketitle

\section{Problem}

Let $\left(\Omega, \mathcal{F}, \mathbb{P}\right)$ be a filtered probability space with the filtration $\mathbb{F} = \left(\mathcal{F}_t\right)_{t \geq 0}$. Let $W = (W_t)_{t \geq 0}$ be a Wiener process on this probability space. Let $0 < a < b$ two positive numbers

We would like to compute the following probability

\begin{equation}
	\mathbb{P}\left(\exists t \in \left[a,b\right], W_t = 0\right)
\end{equation}

\section{Auxiliaries}

\begin{definition}
	Let $x \in \mathbb{R}$. Let $\tau_x$ be the first time when the first time when the Wiener process reaches $x$
	\begin{equation}
		\tau_x = \min\left(t > 0, W_t = x\right)
	\end{equation}
\end{definition}

\begin{proposition}
	The probability density function of $\tau_x$ is 
	\begin{equation}
		f_{\tau_x} (y) = \frac{\left|x\right|}{\sqrt{2 \pi t^3}} \exp\left(-\frac{x^2}{2t}\right)
	\end{equation}
\end{proposition}

\begin{proof}
	\begin{equation*}
	\begin{aligned}
		\mathbb{P}\left(\tau_x < t\right) &= \mathbb{P}\left(\sup_{s \leq t} W_s > x\right)\\
		&= \mathbb{P}\left(\left|W_t\right| > x\right)\\
		&= \mathbb{P}\left(W_t^2 > x^2\right)\\
		&= \mathbb{P}\left(t W_1^2 > x^2\right)\\
		&= \mathbb{P}\left(\frac{x^2}{W_1^2} < t\right)\\
	\end{aligned}
	\end{equation*}
\end{proof}

\begin{corollary}
	$\tau_x$ and $\frac{x^2}{W_1^2}$ have same law
\end{corollary}

\begin{proposition}
\end{proposition}

\end{document}
