\documentclass{article}

\usepackage{amsmath}
\usepackage{amssymb}
\usepackage{dsfont}
\usepackage{enumitem}
\usepackage{amsthm} % to enable the proofs
\usepackage{graphicx} % to enable the graphics
\usepackage{mathtools}

\usepackage{bbm}

\usepackage[hidelinks]{hyperref}

% to be read by all users : how to number theorems and definitions : 
% done by A. Humeau on 4/10/2012
\newtheorem{theorem}{Theorem}[section]
\newtheorem{corollary}[theorem]{Corollary}
\newtheorem{proposition}[theorem]{Proposition}
\newtheorem{lemma}[theorem]{Lemma}

\newtheorem{definition}{Definition}[section]
\newtheorem{remark}{Remark}[section]
\newtheorem{example}{Example}[section]
\newtheorem{notation}{Notation}[section]

% end of environment addition

\usepackage{geometry}
\geometry{top=2cm, bottom=2cm, left=2cm, right=2cm}

\usepackage[utf8]{inputenc}

% to insert code in the report
\usepackage{listings}
\usepackage{xcolor}
\lstset { %
    language=C++,
    backgroundcolor=\color{white}, % set backgroundcolor
    basicstyle=\footnotesize,% basic font setting
}
%end of addition to insert code in the report

\bibliographystyle{plain}

\DeclareMathOperator*{\mysup}{sup}

\begin{document}

\title{Stopping times in mean-reverted processes}
\author{Alexandre Humeau}
\date{\today}
\maketitle

\section{Motivations}
Let us consider a single-dimensional stochastic process $X$ under a probability $\mathbb{P}$. The SDE of $X$ under $\mathbb{P}$ is the following 

\begin{equation}
	\label{eq:GeneralSDE}
	\begin{aligned}
		dX_t  &= \mu\left(t,X_t\right) dt + \sigma\left(t,X_t\right) dW_t\\
		X_0 &= x
	\end{aligned}
\end{equation}

\noindent where $W$ is a Wiener process under $\mathbb{P}$.

\begin{proposition}[Existence and uniqueness of solution]
	Let $T>0$ and $\mu$ and $\sigma$ be measurable functions for which there exists constants $\alpha$ and $\beta$ such that
	\begin{equation}
		\begin{aligned}
			&\left|\mu(t,x\right| + \left|\sigma(t,x)\right| \leq \alpha (1 + \left|x\right|)\\
			&\left|\mu(t,x) - \mu(t,y)\right| + \left|\sigma(t,x) - \sigma(t,y)\right| \leq \beta \left|x-y\right|
		\end{aligned}
	\end{equation}
	if $x$ is a random variable independent of the $\sigma$-algebra generated by $(W_s)_{s \geq 0}$ with finite second moment then (\ref{eq:GeneralSDE}) has a $\mathbb{P}$-almost surely unique solution $(t,\omega) \mapsto X_t(\omega)$ such that X is adapted to the filtration generated by $(W_s)_{s\geq 0}$ and Z and 
	\begin{equation}
		\mathbb{E}^{\mathbb{P}}\left[\int_0^T X_t^2 dt \right] < \infty
	\end{equation}
\end{proposition}

\noindent Let us now assume in the following that all of this conditions are verified.

\noindent Let pose $x \in \mathbb{R}$, we will denote by $\tau_{x,a}$ the hitting time of $a$ for a process beginning in $x$

\begin{equation}
	\tau_{x,a} = \inf\left\{t \leq 0, X_t = a \right\}
\end{equation}

\noindent What is the law of $\tau_{x,a}$ ?

\noindent Let us denote by $u_{x,a}$ the Laplace transform of $\tau_{x,a}$

\begin{equation}
	u_{x,a}(y) = \mathbb{E}^\mathbb{P}\left[e^{-y\tau_{x,a}}\right]
\end{equation}

\section{Ornstein-Ulhenbeck}

Let assume first that $X$ is a Ornstein-Ulhenbeck process. There exist measurable functions $t \mapsto \sigma(t)$ and $t \mapsto \lambda_t$ such that

\begin{equation}
\begin{aligned}
	dX_t &= -\lambda(t) X_t dt + \sigma(t) dW_t\\
	X_0 &= x \in \mathbb{R}
\end{aligned}
\end{equation}

\subsection{Constant coefficients}
Let us now assume that $t \mapsto \sigma(t)$ and $t \mapsto \lambda(t)$ are constant function.

Then
\begin{equation}
	X_t = x e^{-\lambda t} + \sigma e^{-\lambda t} \int_0^t e^{\lambda s} dW_s 
\end{equation}

\noindent From the Dubins-Schwartz theorem, there exist a Wiener process $B = (B_t)_{t \leq 0}$ such that 

\begin{equation}
\begin{aligned}
	B_{\kappa(t)} &= \int_0^t e^{\lambda s} dW_s \\
	\kappa(t) &= \sigma^2 \frac{1 - e^{-2 \lambda t}}{2 \lambda}
\end{aligned}
\end{equation}

\begin{thebibliography}{9}

\bibitem{AliliPatiePedersen}
  L. Alili, P. Patie and J.L. Pedersen
  \emph{Representations of the First Hitting Time Density of an Ornstein-Ulhenbeck Process}

\end{thebibliography}

\end{document}
