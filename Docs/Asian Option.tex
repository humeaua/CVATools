\documentclass{article}

\usepackage{amsmath}
\usepackage{amssymb}
\usepackage{dsfont}
\usepackage{enumitem}
\usepackage{amsthm} % to enable the proofs
\usepackage{graphicx} % to enable the graphics
\usepackage{mathtools}

\usepackage{bbm}

\usepackage[hidelinks]{hyperref}

% to be read by all users : how to number theorems and definitions : 
% done by A. Humeau on 4/10/2012
\newtheorem{theorem}{Theorem}[section]
\newtheorem{corollary}[theorem]{Corollary}
\newtheorem{proposition}[theorem]{Proposition}
\newtheorem{lemma}[theorem]{Lemma}

\newtheorem{definition}{Definition}[section]
\newtheorem{remark}{Remark}[section]
\newtheorem{example}{Example}[section]
\newtheorem{notation}{Notation}[section]

% end of environment addition

\usepackage{geometry}
\geometry{top=2cm, bottom=2cm, left=2cm, right=2cm}

\usepackage[utf8]{inputenc}

% to insert code in the report
\usepackage{listings}
\usepackage{xcolor}
\lstset { %
    language=C++,
    backgroundcolor=\color{white}, % set backgroundcolor
    basicstyle=\footnotesize,% basic font setting
}
%end of addition to insert code in the report

\bibliographystyle{plain}

\DeclareMathOperator*{\mysup}{sup}

\begin{document}

\title{Analytical formula for an Asian option}
\author{Alexandre Humeau}
\date{\today}
\maketitle

\section{Payoff}

Let us consider an Asian european option (call or put). The payoff of such option in $T$ is 

\begin{equation}
	\left(\omega \left( \bar{S} - K\right)\right)^+
\end{equation}

\noindent where

\begin{equation}
	\bar{S} = \frac{1}{n} \sum_{i=1}^n S_{t_i}
\end{equation}
 
\noindent with $0 = t_0 < t_1 < t_2 < \dots < t_n \leq T$.

\section{Model}

Let us assume that the stock price $S$ follows a GBM under the risk neutral probability

\begin{equation}
	\frac{dS_t}{S_t} = rdt + \sigma dW_t
\end{equation}

\noindent Then

\begin{equation}
	\forall i \in [1,n], S_{t_i} = S_0 e^{(r - \frac{\sigma^2}{2})t_i + \sigma W_{t_i}}
\end{equation}

\subsection{Moment matching method}
\subsubsection{Motivations}
As we want to find a formula in the Black-Scholes world for the Asian option, let us try to fit a Black-Scholes lognormal underlying for $\bar{S}$.

\begin{equation}
	\frac{d\bar{S}_t}{\bar{S}_t} = \bar{\mu}dt + \bar{\sigma} dW_t
\end{equation}

\subsubsection{Calculation}

\begin{equation}
\begin{aligned}
	\mathbb{E}\left[\bar{S}\right] &= S_0 e^{\bar{\mu} T}\\
	&= \frac{1}{n} \sum_{i=1}^{n} \mathbb{E}\left[S_{t_i}\right]\\
	&= \frac{1}{n} \sum_{i=1}^{n} S_0 e^{r t_i}
\end{aligned}
\end{equation}

\noindent Then 

\begin{equation}	
	\bar{\mu} = \frac{1}{T} \ln\left(\frac{1}{n} \sum_{i=1}^{n} e^{r t_i}\right)
\end{equation}

\begin{equation}
\begin{aligned}
	Var\left(\bar{S}\right) &= S_0^2 e^{2\bar{\mu}T}\left( e^{\bar{\sigma}^2 T} - 1\right)\\
	&= \frac{1}{n^2} \left(\sum_{i=1}^{n} Var\left(S_{t_i}\right) + 2 \sum_{i < j = 1}^{n} Cov\left(S_{t_i}, S_{t_j}\right)\right)\\
	&= \frac{1}{n^2} \left(\sum_{i=1}^{n} S_0^2 e^{2r t_i} \left(e^{ \sigma^2t_i} -1\right)+ 2 \sum_{i < j = 1}^{n} Cov\left(S_{t_i}, S_{t_j}\right)\right)\\
\end{aligned}
\end{equation}

\noindent where 

\begin{equation}
\begin{aligned}
	Cov\left(S_{t_i}, S_{t_j}\right) &= \mathbb{E}\left[S_{t_i} S_{t_j}\right] - \mathbb{E}\left[S_{t_i} \right] \mathbb{E}\left[S_{t_j}\right]\\
	&= \mathbb{E}\left[ S_{t_i} \mathbb{E}\left[S_{t_j}  \middle | S_{t_i} \right] \right]  - \mathbb{E}\left[S_{t_i} \right] \mathbb{E}\left[S_{t_j}\right]\\
	&= \mathbb{E}\left[ S_{t_i}^2 \mathbb{E}\left[e^{\left(r - \frac{\sigma^2}{2}\right)\left(t_j - t_i\right) + \sigma \left(W_{t_j} - W_{t_i}\right)} \middle | S_{t_i} \right] \right]  - \mathbb{E}\left[S_{t_i} \right] \mathbb{E}\left[S_{t_j}\right]\\
	&= e^{r\left(t_j - t_i\right)} \mathbb{E}\left[S_{t_i}^2\right] - S_0^2 e^{r\left(t_i + t_j\right)}\\
	&= e^{r\left(t_j - t_i\right)} S_0^2 e^{\left(2r + \sigma^2\right)t_i} - S_0^2 e^{r\left(t_i + t_j\right)}\\
	&= S_0^2 e^{r\left(t_i + t_j\right)} \left( e^{ \sigma^2 t_i }  - 1\right)
\end{aligned}
\end{equation}

\noindent Then,

\begin{equation}
	Var\left(\bar{S}\right) =\frac{1}{n^2} \left(\sum_{i=1}^{n} S_0^2 e^{2r t_i} \left(e^{ \sigma^2t_i} -1\right) + 2 \sum_{i<j=1}^{n}  S_0^2 e^{r\left(t_i + t_j\right)} \left(e^{\sigma^2 t_i} - 1\right) \right)
\end{equation}

\noindent Thus,

\begin{equation}
	e^{2\bar{\mu}T}\left( e^{\bar{\sigma}^2 T} - 1\right) = \frac{1}{n^2} \left(\sum_{i=1}^{n} e^{2r t_i} \left(e^{ \sigma^2t_i} -1\right) + 2 \sum_{i<j=1}^{n} e^{r\left(t_i + t_j\right)} \left(e^{\sigma^2 t_i} - 1\right) \right)
\end{equation}

\noindent Then,

\begin{equation}
	\left( e^{\bar{\sigma}^2 T} - 1\right) = \frac{\frac{1}{n^2} \left(\sum_{i=1}^{n} e^{2r t_i} \left(e^{ \sigma^2t_i} -1\right) + 2 \sum_{i<j=1}^{n} e^{r\left(t_i + t_j\right)} \left(e^{\sigma^2 t_i} - 1\right) \right)}{\left(\frac{1}{n} \sum_{i=1}^{n} e^{rt_i}\right)^2}
\end{equation}

\noindent In conclusion

\begin{equation}
	\bar{\sigma}^2 T = \ln\left(1+\frac{\frac{1}{n^2}\sum_{i,j=1}^{n} e^{r \left(t_i+t_j\right)} \left(e^{ \sigma^2\min\left(t_i,t_j\right)} -1\right)}{\left(\frac{1}{n} \sum_{i=1}^{n} e^{rt_i}\right)^2}\right)
\end{equation}

\end{document}
