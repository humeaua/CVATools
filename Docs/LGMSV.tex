\documentclass{article}

\usepackage{amsmath}
\usepackage{amssymb}
\usepackage{dsfont}
\usepackage{enumitem}
\usepackage{amsthm} % to enable the proofs
\usepackage{graphicx} % to enable the graphics
\usepackage{mathtools}

\usepackage{bbm}

\usepackage[hidelinks]{hyperref}

% to be read by all users : how to number theorems and definitions : 
% done by A. Humeau on 4/10/2012
\newtheorem{theorem}{Theorem}[section]
\newtheorem{corollary}[theorem]{Corollary}
\newtheorem{proposition}[theorem]{Proposition}
\newtheorem{lemma}[theorem]{Lemma}

\newtheorem{definition}{Definition}[section]
\newtheorem{remark}{Remark}[section]
\newtheorem{example}{Example}[section]
\newtheorem{notation}{Notation}[section]

% end of environment addition

\usepackage{geometry}
\geometry{top=2cm, bottom=2cm, left=2cm, right=2cm}

\usepackage[utf8]{inputenc}

% to insert code in the report
\usepackage{listings}
\usepackage{xcolor}
\lstset { %
    language=C++,
    backgroundcolor=\color{white}, % set backgroundcolor
    basicstyle=\footnotesize,% basic font setting
}
%end of addition to insert code in the report

\bibliographystyle{plain}

\DeclareMathOperator*{\mysup}{sup}

\begin{document}

\title{Linear Gaussian Markov Model with Stochastic volatility}
\author{Alexandre Humeau}
\date{\today}
\maketitle

\section{The model}

\subsection{Why adding stochastic volatility}

\begin{itemize}
	\item Replication tentative of the swaption smile for pricing of more complex interest-rate derivatives
\end{itemize}

\subsection{Main equations}

Let $f(t,T)$ be the instantaneous forward rate between $t$ and $T$, we assume without loss of any generality under the risk neutral measure $\mathbb{Q}$ (depending on the currency)

\begin{equation}
	\mathrm{d}f(t,T) =\mu_f(t,T) \mathrm{d}t + \sigma_f(t,T) \cdot \left( \sqrt{v} \mathrm{d}W_t \right)
\end{equation}

\noindent where $\cdot$ represents the scalar product and $v$ is a stochastic volatility process that we will define later in the document.

\begin{remark}
	When dealing with multi-currency frameworks, I will write a separate document as it depends highly on the model put on the FX.
\end{remark}

\begin{remark}
	The number of factors $N$ in the rate model is equal to the dimensionality of the considered vectors $v = \left(v_1, \dots, v_N\right)$, $\sigma_f = \left(\sigma_{f,1}, \dots, \sigma_{f,N}\right)$
\end{remark}

\begin{proposition}[Heath-Jarrow-Morton 1992]
	The absence of arbitrage imples that the drift term $\mu_f$ is equal to 
	\begin{equation}
		\mu_f(t,T) = \sum_{i=1}^{N} v_i(t) \sigma_{f,i}(t,T) \int_t^T \sigma_{f,i}(t,u) du
	\end{equation}
\end{proposition}

\begin{remark}
	For a general specification of $\sigma_{f,i} (t, T )$, the dynamics of the forward rate
curve will be path-dependent, which significantly complicates derivatives pricing
and the application of standard econometric techniques
\end{remark}

\begin{proposition}[Ritchken-Sankarasubramanian]
	If the volatility vector of functions can be factored as 
	\begin{equation}
		\sigma_{f,i}(t,T) = \mathcal{P}_n(T-t) e^{-\int_t^T \lambda_u du}
	\end{equation}
	with a regular function $u \mapsto \lambda_u$, and $\tau \mapsto \mathcal{P}(\tau)$ a polynomial function of the variable, then the Heath-Jarrow-Morton model is markovian.
\end{proposition}

\begin{proof}
	Check if all the conditions are reunited, and check the regularity of the function
\end{proof}

\begin{remark}	
	For simplicity in the calibration, we will set the degree of the polynomial to be $1$ and we are left with the following formula, 
	\begin{equation}
		\sigma_{f,i}(t,T) = \left(\alpha_{1,i} + \alpha_{2,i} (T-t) \right) e^{-\lambda_i (T-t)}
	\end{equation}
	we can also consider a term-structure of coefficients and have the follozing formula
	\begin{equation}
		\sigma_{f,i}(t,T) = \alpha_{1,i}(t) e^{-\lambda_i (T-t)}
	\end{equation}
\end{remark}

\section{Stochastic volatility models}
\subsection{Log-normal stochastic volatility}

\subsection{Heston-like stochastic volatility}

\begin{thebibliography}{9}
	\bibitem{TrolleSchwartz2008} A General Stochastic Volatility Model for the Pricing of Interest Rate Derivatives, published by the Oxford University Press on behalf of The Society for Financial Studies
\end{thebibliography}

\end{document}
