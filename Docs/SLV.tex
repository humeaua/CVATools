\documentclass{article}

\usepackage{amsmath}
\usepackage{amssymb}
\usepackage{dsfont}
\usepackage{enumitem}
\usepackage{amsthm} % to enable the proofs
\usepackage{graphicx} % to enable the graphics
\usepackage{mathtools}

\usepackage{bbm}

\usepackage[hidelinks]{hyperref}

% to be read by all users : how to number theorems and definitions : 
% done by A. Humeau on 4/10/2012
\newtheorem{theorem}{Theorem}[section]
\newtheorem{corollary}[theorem]{Corollary}
\newtheorem{proposition}[theorem]{Proposition}
\newtheorem{lemma}[theorem]{Lemma}

\newtheorem{definition}{Definition}[section]
\newtheorem{remark}{Remark}[section]
\newtheorem{example}{Example}[section]
\newtheorem{notation}{Notation}[section]

% end of environment addition

\usepackage{geometry}
\geometry{top=2cm, bottom=2cm, left=2cm, right=2cm}

\usepackage[utf8]{inputenc}

% to insert code in the report
\usepackage{listings}
\usepackage{xcolor}
\lstset { %
    language=C++,
    backgroundcolor=\color{white}, % set backgroundcolor
    basicstyle=\footnotesize,% basic font setting
}
%end of addition to insert code in the report

\bibliographystyle{plain}

\DeclareMathOperator*{\mysup}{sup}

\begin{document}

\title{Stochastic Local Volatility}
\author{Alexandre Humeau}
\date{\today}
\maketitle

\section{The Model}
\subsection{General description}
Let $T$ be a finite time horizon (the final maturity of the deal we want to price, for example)
Let $S$ be an asset of drift $\mu_t$ under the domestic risk-neutral probability measure. We assume the following dynamics under $\mathbb{Q}_d$

\begin{equation}
	\frac{dS_t}{S_t} = \mu_t dt + \sigma(S_t,t) e^{X_t} dW_t
\end{equation}

\noindent where $(S,t) \mapsto \sigma(S,t)$ is the local volatility function (deterministic function of spot and time, calibrated to the vanilla option market at time $0$. We will develop later a methodology to calibrate that function $\sigma$. $X$ is an Ornstein-Ulhenbeck process so that : 

\begin{equation}
	\forall t \in [0,T], \mathbb{E}\left[e^{X_t}\right] = 1
\end{equation}

\noindent The goal of the stochastic volatility would be to calibrate to the forward smile (to be consistent with the 

\subsection{Stochastic volatility part}
Let us assume that $X$ follows the following dynamics under the domestic risk-neutral probability measure

\begin{equation}
\begin{aligned}
	dX_t &= (\theta(t) - \lambda X_t) dt + \sigma_X dW_t^X\\
	X_0 &= 1
\end{aligned}
\end{equation}

\noindent The coefficients are kept constant for now for simplicity.

\noindent We have 

\begin{equation}
	X_t = e^{-\lambda t} + \int_{0}^t \theta(s) e^{-\lambda (t-s)} ds + \int_0^t \sigma_X e^{-\lambda(t-s)} dW_s^X
\end{equation}

\noindent Then,

\begin{equation}
\begin{aligned}
	\exp\left(e^{-\lambda t} + \int_{0}^t \theta(s) e^{-\lambda (t-s)} ds + \frac{1}{2}\int_0^t \sigma_X^2 e^{-2\lambda(t-s)} ds\right) &= 1\\
	e^{-\lambda t} + \int_{0}^t \theta(s) e^{-\lambda (t-s)} ds + \frac{1}{2}\int_0^t \sigma_X^2 e^{-2\lambda(t-s)} ds &= 0
\end{aligned}
\end{equation}

Setting $f(t) = \theta(t) e^{\lambda t}$ it yields

\begin{equation}
\begin{aligned}
	&a(t) \int_0^t f(s) ds + f(t) = b(t)\\
	&a(t)  = \lambda e^{-\lambda t}\\
	&b(t) = \lambda e^{-\lambda t} - \frac{\sigma_X^2}{2} e^{-2\lambda t}
\end{aligned}
\end{equation}

\end{document}
