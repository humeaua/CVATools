\documentclass{article}

\usepackage{amsmath}
\usepackage{amssymb}
\usepackage{dsfont}
\usepackage{enumitem}
\usepackage{amsthm} % to enable the proofs
\usepackage{graphicx} % to enable the graphics
\usepackage{mathtools}

\usepackage{bbm}

\usepackage[hidelinks]{hyperref}

% to be read by all users : how to number theorems and definitions : 
% done by A. Humeau on 4/10/2012
\newtheorem{theorem}{Theorem}[section]
\newtheorem{corollary}[theorem]{Corollary}
\newtheorem{proposition}[theorem]{Proposition}
\newtheorem{lemma}[theorem]{Lemma}

\newtheorem{definition}{Definition}[section]
\newtheorem{remark}{Remark}[section]
\newtheorem{example}{Example}[section]
\newtheorem{notation}{Notation}[section]

% end of environment addition

\usepackage{geometry}
\geometry{top=2cm, bottom=2cm, left=2cm, right=2cm}

\usepackage[utf8]{inputenc}

% to insert code in the report
\usepackage{listings}
\usepackage{xcolor}
\lstset { %
    language=C++,
    backgroundcolor=\color{white}, % set backgroundcolor
    basicstyle=\footnotesize,% basic font setting
}
%end of addition to insert code in the report

\bibliographystyle{plain}

\DeclareMathOperator*{\mysup}{sup}

\begin{document}

\title{Libor Market Model}
\author{Alexandre Humeau}
\date{\today}
\maketitle

\noindent Let us consider a set of dates $(T_i = T_\alpha + (i - \alpha) \delta)_{i=\alpha,\dots,\beta}$ and the Libor associated to the period $\left[T_i, T_{i+1}\right]$

\begin{equation*}
	L_t^i = \frac{1}{\delta} \left(\frac{B\left(t,T_i\right)}{B\left(t,T_{i+1}\right)}-1\right)
\end{equation*}

\noindent The $T_i$-forward probability measure is denoted by $\mathbb{Q}^i$ (associated to the numeraire $B(t,T_i)$.

\begin{proposition}
	$L_t^i$ is a lognormal martingale of volatiliy $\sigma_i(t) = c_i g(T_i-t)$ under $\mathbb{Q}^{i+1}$
\end{proposition}

Which probability can we take to price a product dependent on all the libors ? $\mathbb{Q}^\beta$ the terminal probability.

\begin{equation}
	\frac{dL_t^i}{L_t^i} = \gamma^i(t) dW_t + \text{drift}^i_t dt
\end{equation}

where $\gamma^i$ is a vectorial volatility function such that $\sigma_i(t) = \|\gamma^i(t)\|$ (function from $\mathbb{R}_+$ to $\mathbb{R}^d$) and $W$ a $d$- dimensional Wiener process

\begin{equation*}
	\left.\frac{d\mathbb{Q}^{i+1}}{d\mathbb{Q}^i} \right|_{\mathcal{F}_{T_i}} = \frac{B\left(T_i,T_{i+1}\right)}{B\left(T_i,T_i\right)}\frac{B\left(0,T_i\right)}{B\left(0,T_{i+1}\right)} = \frac{1}{1 + \delta L_{T_i}^i}\frac{B\left(0,T_i\right)}  {B\left(0,T_{i+1}\right)}
\end{equation*}

\begin{equation*}
	\xi_t^i = \mathbb{E}^{\mathbb{Q}^i} \left[\frac{d\mathbb{Q}^{i+1}}{d\mathbb{Q}^i}\middle|\mathcal{F}_t\right] =  \frac{1}{1 + \delta L_{t}^i}\frac{B\left(0,T_i\right)}  {B\left(0,T_{i+1}\right)}
\end{equation*}

\begin{proposition}
	Let $X_t$ be a $\mathbb{Q}^{i+1}$ martingale, then $X_t \xi_t^i$ is a $\mathbb{Q}^{i}$ martingale
\end{proposition}

\noindent We know that $L^{\beta-2}$ is a $\mathbb{Q}^{\beta-1}$ martingale, then $L^{\beta-2} \left(1 + \delta L^{\beta-1}\right)$ is a $\mathbb{Q}^{\beta}$ martingale. Applying Itô's formula yields

\begin{equation*}
	\left(1 + \delta L^{\beta-1}_t\right) dL^{\beta-2} _t + \delta L^{\beta-2}_t dL^{\beta-1}_t + \delta d\langle L^{\beta-2}, L^{\beta-1}\rangle_t
\end{equation*}

\noindent Taking the drifts,

\begin{equation*}
	\left(1 + \delta L^{\beta-1}_t\right) \text{drift}^{\beta-2}_t L^{\beta-2}_t+ \delta \gamma^{\beta-1}(t)\cdot\gamma^{\beta-2}(t)  L^{\beta-1}_t L^{\beta-2}_t= 0
\end{equation*}

\noindent since $\text{drift}_t^{\beta-1} = 0$. Then

\begin{equation}
	\text{drift}^{\beta-2}_t = -\frac{\delta \gamma^{\beta-1}(t)\cdot\gamma^{\beta-2}(t)  L^{\beta-1}_t}{\left(1 + \delta L^{\beta-1}_t\right)}	
\end{equation}

The general formula is then the following

\begin{theorem}
	\begin{equation}
	\begin{aligned}
		\forall k \in [\alpha,\beta], \text{drift}_t^k &= -\delta \sum_{i=k+1}^{\beta-1} \frac{L_t^i \gamma^k(t) \cdot \gamma^i(t)}{1 + \delta L_t^i}\\
		&= -\delta \sum_{i=k+1}^{\beta-1} \frac{L_t^i \rho^{i,k}(t)\sigma^k(t) \sigma^i(t)}{1 + \delta L_t^i}
	\end{aligned}
	\end{equation}
	where $\rho^{i,k}(t)$ is the correlation between libors $i$ and $k$
\end{theorem}

\begin{thebibliography}{9}
	\bibitem{BraceGatarekMusiela}
	A. Brace, D. Gatarek, M. Musiela, \emph{The market model of interest rate dynamics}, 1997, Mathematical Science, 127-155
\end{thebibliography}

\end{document}
